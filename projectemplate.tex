\documentclass{FR16} 

\begin{document}


\maketitle

\tableofcontents
\newpage

\section{Introduction}

\subsection{Overview}
Early detection of infectious viral diseases rapidly in an inexpensive manner is the need of the hour. Clearly, such detection system would find it application in homeland security, clinical diagnosis and public health. It has been estimated that the 95\% \cite{ninety5} of the deaths that happen due to these viral infections (be it covid, AIDS, TB or Malaria) occur in developing countries

Even though government is bearing huge medicare costs, the mortality rate remains, due to delayed diagnosis. What could be reasons for this delayed diagnosis?

\subsection{Literature survey}
 
\subsubsection{Need for antigen based Quantification }
Most detection kits available in the market detect antibodies generated by our body in response to the pathogen. Hence the detection works on those subjects who are at least a week into the disease, the time taken for the body to respond and generate antibodies, while they are still potential transmitters of the disease. 
%Although there seems to be many detection kits available in the market, most of these test kits detect the antibodies that are generated by our body against the virus. Unfortunately, few weeks are required for  the body to generate the antibodies.Prior to the duration these antibodies have been generated, even the infected person would show negative and within these few weeks this infected person shall transmit the disease to others. Irony is that this infected person shall transmit at larger rate and would be more confident as the test results showed negative. This scenario is a small example to prove that quantification methodology based on antigens are preferable to that of antibodies.
Antigen based quantification methodologies like  Transmission Electron Microscopy (TEM) \cite{tem} and Mass Spectroscopy techniques \cite{massspec} although provides accurate description of the virus charge and its size are time consuming, requires skilled operators  and also quite expensive. Other methods like genome detection are virus specific and may land us into problem due to limited range of detection.

Thus, our aim would be to develop an inexpensive yet a portable tool for diagnosis of covid-19 at the point of care (POC).


\subsubsection{Rationale behind our methodology}
In the proposed method, usage of semiconductor theory is explored for identification of virus and its quantification. It is known that the application of Electric field leads to polarization of the medium (solution in which the virus is present) and also the virus particles. Interaction of the virus with the polarity of the medium and the composition of the virus itself decides the strength of polarization. Viruses comprise of proteins and genetic material which in-turn comprises of either RNA or DNA. Phosphodiester bonds present in both of the nucleic acids shall give rise to partial negative charge. The charge on the protein that wraps the nucleic acid genomes is uncertain and can be neutral, positive or negative. Thus the net charge obtained by the virus shall be the cumulative charges of both the protein and the genetic material. The total polarization  can be understood as sum of Electronic polarization, Atomic polarization and Orientation polarization.

We propose use of impedance spectroscopy to take advantage of polarity in molecules that compose virus and derive its IV-CV characteristics.

%This gets reflected in IV-CV  characteristics, impedance spectroscopy, charging discharging of the virus solution etc.%

Lets assume we have a complex biological media consisting of both the virus and the medium. Every virus gives rise to its own signature. Information regarding this signature can only be obtained only if we can de-embed  the electric signals of the medium from the final measured signal. The methodology of de-embedding and obtaining the number of viruses (impurities in the language of semi-conductor field) shall be explained in  the next section.





\newpage
\section{Methodology}
Our approach would be to utilize, IV  characteristics, impedance spectroscopy, %charging discharging of the virus solution%
to develop a micro-controller/FPGA (Field programmable Gate arrays) based technology to detect these viruses quantitatively.

\subsection{Physical interpretation}

\subsubsection{Physical interpretation of our methodology}
When electric field is applied through a medium in which virus particles are suspended, the phase and amplitude of the field changes. The change depends on the concentration of the virus present in the medium. When electric field passes through the medium the separation of charges takes place in the virus. Thus the virus could be considered as a dipole comprising of equal and opposite charges at a separation. Now, the medium with the viruses can be interpreted as dopants (impurities) in a non-intrinsic semi-conductor material. Thus quantifying the viruses now boils down to calculating the dopant concentration by de-embedding the contribution from the reference medium. Even though the concentration of the virus is very small as compared to medium. It still produces a significant change in the electrical parameters which can be attributed to the accumulated charge on the virus as discussed above.

\subsection{Explanation regarding how de-embedding of reference medium is acheived}
The antibodies are conjugated onto a non-reactive surface and then covid-19 antigens are attached and isolated. (The description of this process is given below). This solution is what we term the medium. This is the reference medium because, later the virus is lysed and the change of impedance of the lysed solution  with respect to the  base solution (reference medium) is utilized to quantify the virus.  Two methods can be followed to remove the effect of reference medium

\begin{enumerate}
    \item \textbf{Measure and subtract the effect of the reference} One can measure the impedance spectra of this base solution (reference medium) and subtract its effect to nullify its contribution.
    
    \item \textbf{Washing the medium with a non-conductive solution} We can measure the conductivity of our reference medium and repeatedly wash it with a non-conductive solution  till further washing makes a no significant change. This procedure eliminates any conductive bias present in our reference medium.
\end{enumerate}


\subsection{Modelling the system}
This is the most important step to understand the Electrical behaviour of our system. Even though we shall be to able to quantify virus and achieve our goals, this modelling shall take us that extra mile and make this methodolgy applicable to other viruses. 

For example, Ahmad et al. have performed electrical impedance measurements on Feline immunodeficiency virus (FIV) and
human immunodeficiency virus (HIV). Although they belong to same family and similar in size, upon application of electric field the resistance increased in case of the former and, whereas in case of the later the resistance decreased. This can be interpreted/modelled as follows:

As we discussed earlier, upon application of electrical field, the 
charge separation takes place and thus one of the virus is acting as a diode in the forward bias whereas the other is acting as a reverse biased diode.






\subsection{Workflow}
We have segmented the above methodology into two major blocks(parts) and further sub-steps to fulfill the above methodology and is enumerated below:
\begin{enumerate}
    \item Isolating the covid virus antigens into a solution
    
    \item Performing the Electrical measurements to quantify them
\end{enumerate}

To complete each of these blocks sub-steps required are enumerated below.
\subsubsection{Isolating the covid virus antigens into a solution (medium)}
%The sub-s teps for obtaining the isolated covid virus antigens are enumerated below:
    \begin{enumerate}
        \item \textbf{Preparation/obtaining of virus along with its Quantification:} The virus is obtained from our collaborating biology company along with its quantification.
    
        \item \textbf{Conjugation of antibodies onto a non-reactive surface} Onto a non-reactive surface antibodies are conjugated, which in turn are used to capture the covid-19 antigens. As repeating washings by DPBS solution and centrifuging is involved to remove the supernatant, this non-reactant surface can be a magnetic bead. Then a magnetic seperator can be used to separate the supernatant. Thus, antibodies are conjugated onto a magnetic bead whose dimensions can go into a centrifuge tube. 
        
        \item \textbf{Capture covid-19 antigens using these conjugated magnetic beads:} The covid-19  antigens are captured onto these magnetic beads.
        
        \item \textbf{De-embedding the effect of medium:} As explained earlier, the impedance of this reference solution needs to be measured so that its effects can be negated from the lysed solution to quantify virus. Described above in detail.
        
        
        \item  \textbf{Viral Lysing} The lysing of the virus is to bring retroviral enzymes, membrane phospholipids , capsid proteins and proteins, intracellular ions,  and nucleic acids into solution
        
    \end{enumerate}


\subsection{Performing the Electrical measurements to quantify viruses} Inside an open-ended coaxial cable these virus suspensions shall be loaded. Then, different frequencies from  few mHz to few MHz frequencies shall be used to obtain the frequency response of the virus suspension. As the literature survey suggest, the response shall be maximum near MHz frequencies, the system shall be built which can go up to few MHz. The oscillation voltage shall be 15 to 20 mV peak to peak. Care shall be taken that the coaxial cables self-resonance frequency is much farther away ($\sim$ few 100 MHz) from the afore-mentioned range. 

After loading the virus suspension, IV (current-voltage) and CV (capacitance voltage) characteristics shall be performed. Then an algorithm based on semi conductor theory shall be used to extract the electrical parameters. Once, the quantification procedure is established, it shall be automated to a plug and play mode to quantify the virus using Micro-controllers.





\newpage
\section{Execution and Validation}





\subsection{Facilities, Equipment and Other Resources}





\subsection{Data, Infrastructure and software needs}



\section{Requested Grant details}

\subsection{Table }
Here a few examples of tables and graphs.
\subsection{One shot lab investment}
\begin{center}
\begin{tabular}{c c c c c c}
\arrayrulecolor{Azzurro}
\hline
{\bfseries S.No} & {\bfseries Item Description}& {\bfseries Quantity} & {\bfseries Cost} \\
\hline
1 & centrifuge & 1 & Rs.50,000\\
\hline
2 & Magnetic Separator & 1 & Rs.1,00,000   \\
\hline
3 &Pippettes (1 $\micro$L , 5mL) & 1 & 50,000\\ 
\hline 
4 &Pippette tips & 40 & 1,00,000\\ 
\hline
5 &Distilled water setup(millipore) & 1  & 6,00,000\\
\hline
6 &Impedance spectrometer & 1 & 50,00,000\\
\hline
7 &Fluorescence Microscope &  1 & Rs.10,00,000\\
\hline
 & Total & & 69,00,000
 \end{tabular}
\end{center}

\subsection{Table }
Here a few examples of tables and graphs.
\subsection{Lab consumables}
\begin{center}
\begin{tabular}{c c c c c c}
\arrayrulecolor{Azzurro}
\hline
{\bfseries S.No} & {\bfseries Item Description}& {\bfseries Quantity/month} & {\bfseries Cost/month}& {\bfseries Total(for 2 years)} \\

\hline 
9 &Chemicals & & Rs.1,00,000 & 24,00,000\\   
\hline
10& Qualified work force &  4 & Rs.50,000 & 48,00,000\\
\hline
 & Total & & 72,00,000
\end{tabular}
\end{center}

\subsection{Product Development Phase}
\begin{center}
\begin{tabular}{c c c c c c}
\arrayrulecolor{Azzurro}
\hline
{\bfseries  } & {\bfseries Product Development Phase}& {\bfseries } & {\bfseries }& {\bfseries 30,00,000} \\
\hline
\end{tabular}
\end{center}


% 
%









\section{Introduzione}
Lorem ipsum dolor sit amet, consectetur adipiscing elit, sed do eiusmod tempor incididunt ut labore et dolore magna aliqua. Ut enim ad minim veniam, quis nostrud exercitation ullamco laboris nisi ut aliquip ex ea commodo consequat. Duis aute irure dolor in reprehenderit in voluptate velit esse cillum dolore eu fugiat nulla pariatur. Excepteur sint occaecat cupidatat non proident, sunt in culpa qui officia deserunt mollit anim id est laborum.
\newpage

\section{Tabelle e grafici}
Here a few examples of tables and graphs.
\subsection{Tabelle}
\begin{center}
\begin{tabular}{c c c c c c c c}
\arrayrulecolor{Azzurro}
\hline
{\bfseries $Codice$} & {\bfseries $CdL$} & {\bfseries $Lotto$} & {\bfseries $T_{setup/lotto}$} & {\bfseries $T_{lav/pezzo}$} & {\bfseries $T_{proc/pezzo}$} & {\bfseries$Quantit\grave{a}$} & {\bfseries $T_{tot}$}\\
\hline
100 & 4 & 250 & 25 & 0,5 & 0,6 & 1 & 0,6\\
111 & 2 & 250 & 20 & 2 & 2,08 & 1 & 2,08 \\
111 & 3 & 250 & 15 & 1,5 & 1,56 & 1 & 1,56 \\
112 & 2 & 250 & 20 & 2,5 & 2,58 & 1 & 2,58 \\
112 & 3 & 250 & 15 & 2 & 2,06 & 1 & 2,06\\
113 & 3 & 500 & 15 & 1 & 1,03 & 2 & 2,06\\
120 & 1 & 50 & 30 & 2 & 2,6 & 0,1 & 0,26\\
121 & 1 & 25 & 30 & 3 & 4,2 & 0,1 & 0,42 \\
121 & 1 & 25 & 30 & 2,5 & 3,7 & 0,1 & 0,37 \\
\hline
\end{tabular}
\end{center}

\subsubsection{Altra tabella}
\begin{center}
\begin{tabular}{l c c c c c c c c c}
\arrayrulecolor{Azzurro}
\hline
{\bfseries Periodo} & 1 & 2 & 3 & 4 & 5 & 6 & 7 & 8 & Media\\
\hline
{\bfseries MPS} & 250 & 250 & 250 & 250 & 250 & 250 & 250 & 250 & \\
{\bfseries CdL 1} & 262,5 & 262,5 & 262,5 & 262,5 & 262,5 & 262,5 & 262,5 & 262,5 & 262,5\\
{\bfseries CdL 2} & 1165 & 1165 & 1165 & 1165 & 1165 & 1165 & 1165 & 1165 & 1165\\
{\bfseries CdL 3} & 1420 & 1420 & 1420 & 1420 & 1420 & 1420 & 1420 & 1420 & 1420\\
{\bfseries CdL 4} & 150 & 150 & 150 & 150 & 150 & 150 & 150 & 150 & 150\\
\hline
\end{tabular}
\end{center}

\subsection{Grafici}
\begin{center}
\begin{figure}[H]
\begin{tikzpicture}
\begin{axis}[
/pgf/number format/.cd,
        use comma,
        1000 sep={},
ybar,
ymin=0,ymax=2000,
ymajorgrids=true,
ylabel=,
xlabel=,
major x tick style = transparent,
enlarge x limits=0.1,
legend style={draw=none,font=\scriptsize,cells={anchor=west}},
symbolic x coords={1,2,3,4,5,6,7,8},
bar width=5pt, %!!!!!!!!!!!!!!!!
xtick=data,
nodes near coords={
    \rotatebox{90}{%
    \tiny
    \pgfmathprintnumber[fixed,precision=0,zerofill]\pgfplotspointmeta
    }
},
nodes near coords align={vertical},
width=13cm,
height=8cm,
legend style={
    legend pos=outer north east,
    legend columns=1,
}
]
\addplot[draw=Rosso, fill=Rosso] coordinates    {(1,262.5) (2,262.5) (3,262.5) (4,262.5) (5,262.5) (6,262.5) (7,262.5) (8,262.5)};    
\addplot [draw=Viola, fill=Viola] coordinates
    {(1,1165) (2,1165) (3,1165) (4,1165) (5,1165) (6,1165) (7,1165) (8,1165)};    
\addplot [draw=Arancione, fill=Arancione] coordinates
    {(1,1420) (2,1420) (3,1420) (4,1420) (5,1420) (6,1420) (7,1420) (8,1420)};    
    \addplot [draw=Celeste, fill=Celeste] coordinates
    {(1,150) (2,150) (3,150) (4,150) (5,150) (6,150) (7,150) (8,150)}; 

\legend{CdL 1, CdL 2, CdL 3, CdL 4}
\end{axis}
\end{tikzpicture}
\end{figure}
\end{center}

\subsubsection{Altro grafico}
\begin{center}


\begin{tikzpicture}

    
        
    \draw[->,name path=xaxis] (-0.2,0) -- (10.2,0) node[right] {$t$};
    \draw[->,name path=yaxis] (0,-0.8) -- (0,3.2) node[above] {Livello $x(t)$};

    % lines  
    \draw[name path=line1,domain=0:5] plot (\x,{2-2/3* \x}) node[above right] {};
    \draw[name path=line2,domain=-0.666:2] plot ({4},\x) node[below right] {};
    \draw[name path=line3,domain=4:8] plot (\x,{14/3-2/3* \x}) node[above right] {};
    \draw[name path=line4,domain=0:9,  draw=gray] plot (\x,{2}) node[above right] {};
    \draw[name path=line5,domain=0:9,  draw=gray] plot (\x,{-0.666}) node[above right] {};
\draw[name path=line6,domain=0:-1.7,  draw=gray] plot ({0}, \x) node[above right] {};
    \draw[name path=line7,domain=0:-1.7,  draw=gray] plot ({4},\x) node[above right] {};
    \draw[name path=line8,domain=0:-1.1,  draw=gray] plot ({3},\x) node[above right] {};
    
    % calculate intersection points
    \node[name intersections={of=line1 and line2}] (a) at (intersection-1)  {};
    \node[name intersections={of=line2 and line3}] (b) at (intersection-1) {};
    \node[name intersections={of=line1 and xaxis}] (c) at (intersection-1) {};
    \node (d) at (0,0) {};
    \node[name intersections={of=yaxis and line1}] (e)  at (intersection-1){};
    \node[name intersections={of=line2 and xaxis}] (f) at (intersection-1) {};
    \node[name intersections={of=line3 and xaxis}] (g) at (intersection-1) {};
    
    % draw the big polygon    
    \filldraw[fill=Azzurro,fill opacity=0.4] (c.center) -- (d.center) -- (e.center) -- cycle;
\filldraw[fill=Blu,fill opacity=0.7] (f.center) -- (a.center) -- (c.center) -- cycle;
   \filldraw[fill=Azzurro,fill opacity=0.4] (g.center) -- (f.center) -- (b.center) -- cycle; 
   
  \draw [<->] (9,1.9) -- (9,-0.566); 
  \node[align=left, right] at (9,0.667) {$Q$}; 
  \draw [<->] (8.2,-0.1) -- (8.2,-0.566); 
  \node[align=left, right] at (8.2,-0.333) {$B$};
  
  \draw [<->] (0.1,-1.7) -- (3.9,-1.7); 
  \node[] at (2,-1.95) {$T$};
  \draw [<->] (0.1,-1.1) -- (2.9,-1.1); 
  \node[] at (1.5,-1.35) {$t_{i}$};
  \draw [<->] (3.1,-1.1) -- (3.9,-1.1); 
  \node[] at (3.5,-1.35) {$t_{b}$};
  
 \draw [] (4.872,-0.666) arc [start angle=0, end angle=-30, radius=1cm]
    node [midway, right] {$r$};   
 
\end{tikzpicture}
\end{center}
\newpage

\section{Formule}
Se non sono ammesse consegne in ritardo siamo in presenza di un problema con Backlog. Sia $ t_{i} $ il periodo in cui non si è in backlog e $ t_{b} $ il periodo di backlog. Essendo $ t_{i}=(Q-B)/D $, avremo:\\
Costi di ordinazione = $C\cdot D/Q$\\
Costi di mantenimento = $ H\cdot (Q-B)/2\cdot t_{i}/T=H\cdot (Q-B)^{2}/2Q $\\
Costi di backorder = $ C_{b}\cdot B\cdot t_{b}/2T=C_{b}\cdot B^{2}/2Q $\\
Costi variabili totali = $ TC(Q)=C\cdot D/Q+H\cdot (Q-B)^{2}/2Q+C_{b}\cdot B^{2}/2Q $\\
Condizioni di minimo: 
$\begin{cases}
\frac{\partial TC}{\partial Q}=0\\\frac{\partial TC}{\partial B}=0
\end{cases}
\Rightarrow
Q^{\ast}=\sqrt{\dfrac{2C\cdot D(H+C_{b})}{H\cdot C_{b}}}=EOQ\sqrt{\dfrac{H+C_{b}}{C_{b}}}
$

\section{Altro}
\begin{figure}[H]
\centering
\includegraphics[width=1\textwidth]{grafo.png}
\caption{\label{fig:1}Didascalia.}
\end{figure}
\subsection{Footnote}
You can create a footnote like this.\footnote{I created a footnote.}

\subsection{Flowchart}
\newpage
\begin{landscape}
\thispagestyle{empty}
\singlespacing
\noindent
\begin{tikzpicture}[overlay,remember picture,node distance = 3.5cm, auto]
\coordinate (0) at (current page.north);
\tikzstyle{decision} = [diamond,aspect=2, draw=Blu,line width=0.6mm, fill=Blu!10, minimum height=1em,
    text width=6em, text badly centered,  inner sep=0pt]
\tikzstyle{block} = [rectangle, draw=Blu,line width=0.6mm, fill=white, text width=5em, text badly centered, rounded corners, minimum height=1em]
\tikzstyle{line} = [draw, -latex']
\tikzstyle{cloud} = [draw=white, ellipse,fill=Blu, node distance=1.5cm,
    minimum height=2em]
    
    % Place nodes
    
    \node [decision][shift={(19.5 cm,-8cm)}]  at (current page.north) (init) {\scriptsize Prodotto standard?};
    \node [cloud, above of=init] (system) {\scriptsize\textcolor{white}{Ricezione ordine dal cliente}};
    \node [decision, right of=init,node distance=4.5cm] (registrosi) {\scriptsize Il cliente accetta di ordinare almeno un cartone di prodotto?};
    \node [decision, right of=registrosi,node distance=4.8cm] (100) {\scriptsize Cambia prodotto?};
    \node [block, right of=100,node distance=3.5cm] (101) {\scriptsize Ciao!};
    \node [decision, below of=init, node distance=2cm] (identify) {\scriptsize  Nuovo cliente?};
    \node [block, below of=identify, node distance=1.7cm] (102) {\scriptsize  Compilo foglio di lavoro};
    \node [block, below of=102, node distance=1.7cm] (103) {\scriptsize  L'operatore prende il foglio di lavoro};
    \node [block, below of=103, node distance=1.9cm] (104) {\scriptsize  Va in magazzino e prende le scatole necessarie};
   \node [decision, below of=104, node distance=2.2cm] (105) {\scriptsize  Le scatole sono rovinate?};
   \node [block, below of=105, node distance=2.4cm] (106) {\scriptsize  Metto le scatole nel cartone per l'imballaggio}; 
   \node [decision, below of=106, node distance=2.2cm] (107) {\scriptsize  Manca qualche articolo?};
   \node [block, right of=107, node distance=4.5cm] (108) {\scriptsize  Sposto il cartone in zona di attesa};
   \node [block, above of=108, node distance=1.5cm] (109) {\scriptsize  Riporto il foglio di lavoro in ufficio};
   \node [block, above of=109, node distance=3.5cm] (110) {\scriptsize  Contatto cliente e avviso sui nuovi tempi di consegna};
   \node [decision, right of=110, node distance=4cm] (111) {\scriptsize  Il cliente è disponibile all'attesa?};
   \node [block, right of=111, node distance=4cm] (112) {\scriptsize  Propongo articolo similare disponibile in magazzino};
   \node [decision, above of=112, node distance=2.5cm] (113) {\scriptsize  Va bene?};
   \node [decision, above of=113, node distance=2.7cm] (114) {\scriptsize  Il cliente vuole comunque gli altri articoli?};
   \node [block, above of=114, node distance=2.2cm] (115) {\scriptsize  Aggiorno il foglio di lavoro};
   \node [block, left of=114, node distance=4.5cm] (116) {\scriptsize  Annullo ordine};
   \node [block, below of=111, node distance=2.3cm] (117) {\scriptsize  Contatto fornitore e ordino gli articoli mancanti};
   \node [block, below of=117, node distance=1.5cm] (118) {\scriptsize  Arrivo merce};
   \node [block, below of=118, node distance=1.3cm] (119) {\scriptsize  Smisto in magazzino};
   \node [block, below of=119, node distance=2cm] (120) {\scriptsize  Aggiungo gli astucci mancanti nel cartone per l'imballaggio};
   \node [block, right of=118, node distance=6cm] (121) {\scriptsize  Aggiorno il foglio di lavoro};
   \node [block, left of=identify, node distance=3.5cm] (122) {\scriptsize  Registro anagrafica cliente};
   \node [decision, left of=122, node distance=4cm] (123) {\scriptsize  Prodotto personalizzato?};
   \node [block, above of=123, node distance=2.4cm] (124) {\scriptsize  Ricevo informazioni sul clichè};
    \node [block, left of=124, node distance=3.5cm] (125) {\scriptsize  Contatto il disegnatore per realizzare il clichè};
    \node [block, below of=125, node distance=2.4cm] (126) {\scriptsize  Mando al cliente prima bozza disegno};
    \node [decision, below of=126, node distance=3.5cm] (127) {\scriptsize  Cliente soddisfatto?};
    \node [block, below of=127, node distance=1.5cm] (128) {\scriptsize  Nuova bozza};
    \node [block, right of=127, node distance=3.5cm] (129) {\scriptsize  Creazione clichè};
    \node [block, below of=129, node distance=1cm] (130) {\scriptsize  Ricezione clichè};
    \node [block, right of=130, node distance=3.5cm] (131) {\scriptsize  Metto clichè nel cassetto di raccolta};
    \node [block, left of=105, node distance=4cm] (132) {\scriptsize  Apro e controllo gli articoli};
    \node [decision, left of=132, node distance=4cm] (133) {\scriptsize  Anche gli astucci sono rovinati?};
    \node [block, left of=133, node distance=4cm] (134) {\scriptsize  Cambio solo la scatola mantenendo gli astucci};
    \node [block, below of=133, node distance=2.4cm] (135) {\scriptsize  Prendo nuova scatola};
    \node[below  = 2.5cm of 107](136){};
    
   
   
   
   
   
       
\scriptsize     
    % Draw edges
    \path [line,name path=first] (init) -- node {no}(identify);
    \path [line] (init) -- node {sì}(registrosi);
    \path [line] (system) -- (init);
    \path [line] (registrosi) -- node{no}(100);
    \path [line] (100) -- node{no}(101);
    \path [line,name path=registrositoidentify] (registrosi)    |- node [near start] {sì}  ([xshift=0cm, yshift=0cm]identify.east)  (identify);
    \path [line,name path=100toidentify] (100)    |- node [near start] {sì}  ([xshift=0cm, yshift=0cm]identify.east)  (identify);
    \path [line] (identify) -- node{no}(102);
    \path [line] (102) -- (103);
    \path [line] (103) -- (104);
    \path [line] (104) -- (105);
    \path [line] (105) -- node{no}(106);
    \path [line] (106) -- (107);
    \path [line] (107) -- node{sì}(108);
    \path [line] (108) -- (109);
    \path [line] (109) -- (110);
    \path [line] (110) -- (111);
    \path [line] (111) -- node{no}(112);
    \path [line] (112) -- (113);
    \path [line] (113) -- node{no}(114);
    \path [line] (114) -- node{sì}(115);
    \path [line] (114) -- node{no}(116);
    \path [line] (111) -- node{sì}(117);
    \path [line] (117) -- (118);
    \path [line] (118) -- (119);
    \path [line] (119) -- (120);
    \path [line] (113) -| node{sì}(121);
    \path [line] (121) |- (120);
    \path [line] (identify) -- node{sì}(122);
    \path [line] (122) -- (123);
    \path [line] (123) -- node{sì}(124);
    \path [line] (124) -- (125);
    \path [line] (125) -- (126);
    \path [line] (123) |- node [near start] {no}(102);
    \path [line] (126) -- (127);
    \path [line] (127) -- node{no} (128);
    \path [line] (128) -|  ([xshift=-0.3cm, yshift=0cm]127.west) |- ([xshift=0cm, yshift=0cm]126.west) (126);
    \path [line] (127) -- node{sì} (129);
    \path [line] (129) -- (130);
    \path [line] (130) -- (131);
    \path [line] (131) --  ([xshift=0cm, yshift=0cm]131.north) |- ([xshift=0cm, yshift=-0.2cm]102.west) (102);
    \path [line] (105) -- node{sì} (132);
    \path [line] (132) -- (133);
    \path [line] (133) -- node{no} (134);
    \path [line] (133) -- node{sì} (135);
    \path [line] (135) -- (106);
    \path [line] (134) --  ([xshift=0cm, yshift=0cm]134.south) |- ([xshift=0cm, yshift=-0.7cm]106.west) (106);
    \path [line] (107) -- node{no} (136);
    \path [line] (115) -| ([xshift=0.2cm, yshift=0cm]121.east) |- ([xshift=0cm, yshift=0.2cm]136.east) (136);
    \path [line] (120) |- ([xshift=0cm, yshift=0.3cm]136.east) (136);
   
    
    
    
    

\end{tikzpicture}

\end{landscape}
\newpage
\begin{landscape}
\thispagestyle{empty}
\singlespacing
\noindent
\begin{tikzpicture}[overlay,remember picture,node distance = 3.5cm, auto]
\coordinate (0) at (current page.north);
\tikzstyle{decision} = [diamond,aspect=2, draw=Blu,line width=0.6mm, fill=Blu!10, minimum height=1em,
    text width=6em, text badly centered,  inner sep=0pt]
\tikzstyle{block} = [rectangle, draw=Blu,line width=0.6mm, fill=white, text width=5em, text badly centered, rounded corners, minimum height=1em]
\tikzstyle{line} = [draw, -latex']
\tikzstyle{cloud} = [draw=white, ellipse,fill=Blu, node distance=1.5cm,
    minimum height=2em]
    
    % Place nodes
    \node [shift={(19.5 cm,-5.5cm)}]  at (current page.north) (1) {};
    \node [decision, below of=1, node distance=2cm] (2) {\scriptsize Prodotto personalizzato?};
    \node [block, below of=2, node distance=3cm] (3) {\scriptsize Aggiungo nel cartone per imballaggio i cieli su cui effettuare la stampa};
    \node [block, below of=3, node distance=2.2cm] (4) {\scriptsize Metto il cartone in zona attesa};
    \node [decision, below of=4, node distance=2cm] (5) {\scriptsize Il clichè è presente nel cassetto di raccolta?};
    \node [block, below of=5, node distance=2.1cm] (6) {\scriptsize Clichè ok sul foglio di lavoro};
    \node [block, right of=6, node distance=3cm] (7) {\scriptsize Operatore 1: apro gli astucci};
    \node [block, right of=7, node distance=3cm] (8) {\scriptsize Incollo i cieli};
    \node [decision, right of=8, node distance=3.2cm] (9) {\scriptsize Errore di incollaggio?};
    \node [block, right of=9, node distance=3.5cm] (10) {\scriptsize Richiudo gli astucci};
    \node [block, above of=10, node distance=1.5cm] (11) {\scriptsize Metto gli astucci nella scatola};
    \node [block, above of=11, node distance=1.8cm] (12) {\scriptsize Inserisco la scatola nel cartone per imballaggio};
    \node [block, below of=7, node distance=3cm] (13) {\scriptsize Operatore 2: metto il clichè sulla macchina};
    \node [block, right of=13, node distance=3cm] (14) {\scriptsize Prendo i cieli da stampare};
     \node [block, right of=14, node distance=3cm] (15) {\scriptsize Stampo};
     \node [decision, right of=15, node distance=3.2cm] (16) {\scriptsize Stampa ok?};
     \node [block, below of=16, node distance=2cm] (17) {\scriptsize Prendo i cieli mancanti in magazzino};
     \node [block, left of=2, node distance=4cm] (18) {\scriptsize Riporto in ufficio il foglio di lavoro};
     \node [block, left of=18, node distance=3cm] (19) {\scriptsize Contatto cliente per somma contrassegno e avviso spedizione};
     \node [block, left of=19, node distance=3cm] (20) {\scriptsize Genero il talloncino per la spedizione};
      \node [block, below of=20, node distance=1.5cm] (21) {\scriptsize Imballo il cartone};
      \node [block, below of=21, node distance=1cm] (22) {\scriptsize Attesa del corriere};
      \node [block, left of=5, node distance=4cm] (23) {\scriptsize Contatto il cliente per il clichè da realizzare};
      \node [decision, left of=23, node distance=4.2cm] (24) {\scriptsize Il cliente è disposto ad aspettare che venga realizzato il clichè?};
      \node [decision, above of=24, node distance=2.7cm] (25) {\scriptsize Accetta articoli senza personalizzazione?};
      \node [block, above of=23, node distance=1.3cm] (26) {\scriptsize Annulla ordine};
      \node [block, above of=26, node distance=2.7cm] (27) {\scriptsize Elimino i cieli dal cartone};
      \node [block, above of=27, node distance=1.5cm] (28) {\scriptsize Comunico importo};
      \node [block, left of=28, node distance=3cm] (29) {\scriptsize Aggiorno foglio lavoro};
      \node [block, below of=24, node distance=4cm] (125) {\scriptsize  Contatto il disegnatore per realizzare il clichè};
    \node [block, below of=125, node distance=2.2cm] (126) {\scriptsize  Mando al cliente prima bozza disegno};
    \node [decision, right of=126, node distance=4.2cm] (127) {\scriptsize  Cliente soddisfatto?};
    \node [block, below of=127, node distance=1.5cm] (128) {\scriptsize  Nuova bozza};
    \node [block, above of=127, node distance=1.5cm] (129) {\scriptsize  Creazione clichè};
    \node [block, above of=129, node distance=1cm] (130) {\scriptsize  Ricezione clichè};
    \node [block, below of=23, node distance=2.1cm] (131) {\scriptsize  Metto clichè nel cassetto di raccolta};
    
    
    
    
\scriptsize      
    \path [line] (1) -- (2);
    \path [line] (2) -- node{sì} (3);
    \path [line] (3) -- (4);
    \path [line] (4) -- (5);
    \path [line] (5) -- node{sì} (6);
    \path [line] (6) -- (7);
    \path [line] (7) -- (8);
    \path [line] (8) -- (9);
    \path [line] (9) -- node{no} (10);
    \path [line] (10) -- (11);
    \path [line] (11) -- (12);
    \path [line] (6) |- (13);
    \path [line] (13) -- (14);
    \path [line] (14) -- (15);
    \path [line] (15) -- (16);
    \path [line] (16) -- node{no} (17);
    \path [line] (17) -| (15);
    \path [line] (16) |- node [near start] {sì} ([xshift=0cm, yshift=0.5cm]16.north) -| ([xshift=0cm, yshift=0cm]8.south) (8);
    \path [line] (9) |- node [near start] {sì} ([xshift=0cm, yshift=0.8cm]16.north) -| ([xshift=0.5cm, yshift=0cm]16.east) |- ([xshift=0cm, yshift=0cm]17.east) (17);
    \path [line] (2) -- node [near start] {no} (18);
    \path [line] (12) -|  ([xshift=0.8cm, yshift=0cm]11.east) |- ([xshift=0cm, yshift=-2cm]17.south) -| ([xshift=-10.5cm, yshift=0cm]3.west)  |- ([xshift=0cm, yshift=1cm]18.north) -| (18);
    \path [line] (18) -- (19);
    \path [line] (19) -- (20);
    \path [line] (20) -- (21);
    \path [line] (21) -- (22);
    \path [line] (5) -- node{no} (23);
    \path [line] (23) -- (24);
    \path [line] (24) -- node{no} (25);
    \path [line] (25) -| node  {no} (26);
    \path [line] (25) |- node [near start]  {sì} ([xshift=0cm, yshift=0.2cm]25.north)  --   (27);
    \path [line] (27) -- (28);
    \path [line] (28) -- (29);
    \path [line] (29) -|  ([xshift=-0.2cm, yshift=-0.2cm]19.west) |- ([xshift=0cm, yshift=-0.2cm]20.east) (20);
    \path [line] (24) -- node{sì} (125);
    \path [line] (125) -- (126);
    \path [line] (126) -- (127);
    \path [line] (127) -- node{no} (128);
    \path [line] (127) -- node{sì} (129);
    \path [line] (129) -- (130);
    \path [line] (130) -- (131);
    \path [line] (131) -- (6);
    \path [line] (128) -| (126);
    
    
    

\end{tikzpicture}

\end{landscape}
\newpage



\newpage
\begin{thebibliography}{9}
\bibitem{ninety5}
P. Yager , G. J. Domingo , J. Gerdes , Annu. Rev. Biomed. Eng. 2008 ,
10 , 107
\bibitem{massspec}
Uetrecht, C. et al. Interrogating viral capsid assembly with ion mobility–mass
spectrometry. Nature Chem. 3, 126–132 (2011).
\bibitem{tem}
Roingeard, P. Viral detection by electron microscopy: past, present and future.
Biol. Cell. 100, 491–501 (2008).
\bibitem{sandvik}
Sandvik Coromant,\emph{Catalogo  generale  2018},   http://www.coromant.sandvik.com/it
\bibitem{uni}
Norme UNI, Ente nazionale italiano di unificazione
\end{thebibliography}

\end{document}